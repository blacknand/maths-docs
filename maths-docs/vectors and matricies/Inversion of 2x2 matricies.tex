\documentclass[options]{article}
\title{Inversion of 2x2 Matricies}
\author{Nathan Blackburn {@blacknand}}
\date{September 2024}

% Document config
\usepackage{amsmath}
\setlength{\parskip}{1em}  

\begin{document}
\maketitle
Notes on the inversion of 2x2 matrices, which can be applied to an arbitrary number of \( n \cdot n \) matrices.
\vspace{-1em}
\section{The inversion of a matrix}
Just like every number has a reciprocal such as \(8^{-1}\), a matrix \(n\) also has a reciprocal: \(n^{-1}\). When a matrix
is multiplied by it's reciprocal (\(n \cdot n^{-1}\)) the result is \(I\), also known as the identity matrix. The identity matrix
is the matrix equivelance of the number 1 obtained from the reciprocal of a real number, in other words, the identity matrix is just 1, exactly the same
way that the reciprocal of \(\frac{1}{8} \cdot 8 = 1\). This information is not terribly important, but is nontheless useful to know.
\par


A good example of an identity matrix is a \( 3 \times 3 \) matrix, \( I \):

\begin{center}

    \(
        i = 
        \begin{pmatrix}
        1 & 0 & 0\\
        0 & 1 & 0\\
        0 & 0 & 1
        \end{pmatrix}
    \)

\end{center}

You can see here that it has got 1 going down diagonally and 0 in every other row and column. An identity
matrix, as mentioned in the title can be \(2 \cdot 2\), \(3 \cdot 3\), \(4 \cdot 4\)... The inverse of a matrix \(a\) is \(a^{-1}\) only when \(a \cdot a^{-1} = a^{-1} \cdot a = i\)

\section{2x2 matricies}


\end{document}