\documentclass[options]{article}
\title{Inversion of 2x2 Matricies}
\author{Nathan Blackburn {@blacknand}}
\date{September 2024}

% Document config
\setlength{\parskip}{1em}  

\begin{document}
\maketitle
Notes on the inversion of 2x2 matrices, which can be applied to an arbitrary number of \( n \cdot n \) matrices.
\vspace{-1em}
\section{The inversion of a matrix}
Just like every number has a reciprocal such as \(8^-1\), a matrix \(n\) also has a reciprocal: \(n^-1\). When a matrix
is multiplied by it's reciprocal (\(n \cdot n^-1\)) the result is \(I\), also known as the identity matrix. The identity matrix
is the matrix equivelance of the number 1 obtained from the reciprocal of a real number, in other words, the identity matrix is just 1, exactly the same
way that the reciprocal of \(\frac{1}{8} \cdot 8 = 1\).
\par

\end{document}