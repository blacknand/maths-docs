\documentclass[options]{article}
\title{Multiplication, addition and subtraction of matricies}
\author{Nathan Blackburn {blacknand@github}}
\date{September 2024}

% Document config
\usepackage{amsmath}                            % For matricies
\usepackage{enumitem}                           % For lists
\setlength{\parskip}{1em}  

\begin{document}
\maketitle
Notes on basic arithmatical operations on matricies including addition, subtraction and multiplication of different sized matricies.
\section{Multiplication}
Two matricies can only be multiplied together if the number of the columns in the first is the same as the number of rows in the second. The
result of two is a matrix that has the same number of rows as the first matrix, and same number of columns as the second.
\subsection{Example 1}
Find \(ab\) where \(a = \begin{pmatrix} 1 & 4\\6 & 3 \end{pmatrix}\) and \(b = \begin{pmatrix} 2 \\ 5 \end{pmatrix}\)
\begin{center}
    \( 
        c =
        \begin{pmatrix}
            1 & 4 \\ 6 & 3
        \end{pmatrix} 
        \cdot
        \begin{pmatrix}
            2 \\ 5
        \end{pmatrix}
        =
        \begin{pmatrix}
            1 * 2 + 4 * 5\\
            6 * 2 + 3 * 5
        \end{pmatrix}
        =
        \begin{pmatrix}
            22 \\ 27
        \end{pmatrix}
    \)
\end{center}
The first row of \(a\) multiplies the first column of \(b\) to give:
\begin{center}
    \(
        (1 * 2) + (4 * 5) = 2 + 20 = 22
    \)
\end{center}
The second row of \(a\) then multiples by the first column of \(b\) to give:
\begin{center}
    \(
        (6 * 2) + (3 * 5) = 12 + 15 = 27
    \)
\end{center}
\subsection{Example 2}
Find, if possible: \( \begin{pmatrix} 1 & 4 & 9\\2 & 0 & 1 \end{pmatrix} \cdot \begin{pmatrix} 1 & 9\\8 & 7\\-7 & 3 \end{pmatrix} \)

\begin{center}
    \(
        \begin{pmatrix}
            1 \times 1 + 4 \times 8 + 9 \times -7 & 1 \times 9 + 4 \times 7 + 9 \times 3 \\
            2 \times 1 + 0 \times 8 + 1 \times -7 & 2 \times 9 + 0 \times 7 + 1 \times 3
        \end{pmatrix}
    \)
\end{center}
The first row multiplies by both columns, and then the second row multiplies by both columns.
\begin{center}
    \(
        =
        \begin{pmatrix}
            -30 & 64 \\
            -5 & 21
        \end{pmatrix}
    \)
\end{center}

\end{document}