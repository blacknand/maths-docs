\documentclass[options]{article}
\title{Solving two simultaneous unkowns using matricies}
\author{Nathan Blackburn {blacknand@github}}
\date{September 2024}

% Document config
\usepackage{amsmath}                            % For matricies
\usepackage{enumitem}                           % For lists
\setlength{\parskip}{1em}  

\begin{document}
\maketitle
Notes on solving two simultaneous unkown equations using matricies. You must first understand the inversion of 2x2 matricies before reading these notes.

\section{The steps}
Throughout this article, we will be using the example equations: \(3x + 2y = -3\) and \(5x + 3y = -4\)
\begin{enumerate}
    \item Rewrite the equations using matricies:
        \begin{center}
            \(
                \begin{pmatrix}
                    3 & 2\\
                    5 & 3
                \end{pmatrix}
                \cdot
                \begin{pmatrix}
                    x \\y
                \end{pmatrix}
                =
                \begin{pmatrix}
                    -3\\
                    4
                \end{pmatrix}
            \)
        \end{center}
        You can see that the numbers from the first equation, \(3x + 2y\) are in the first row of the matrix, and the numbers
        from the second equation, \(5x + 3y\) are in the second row of the equation. The dot product of these two matricies now result
        in the matrix of the results of both unkown equations, -3 and 4. We now clarify with the following:
        \begin{center}
            \(
                a = 
                \begin{pmatrix}
                    3 & 2\\
                    5 & 3
                \end{pmatrix}
                x =
                \begin{pmatrix}
                    x \\y
                \end{pmatrix}
                b =
                \begin{pmatrix}
                    -3\\
                    4
                \end{pmatrix}
            \)
        \end{center}
    \item Second item:
    
    Now we have \(ax = b\). If \(ax = b\) then we need to find \(x\), to do this we can multiply both sides by the inverse of \(a\) so \(a^{-1}\), also
    remeber that the inverse of \(a \cdot a^{-1} = i\), the inverse matrix which is just 1. 
    \begin{center}
        \(
            a
            \cdot
            a^{-1}
            \cdot
            x
            =
            b\cdot
            a^{-1}
        \)

        \(
            ix = b\cdot a^{-1}
        \)

        \(
            x = b\cdot a^{-1}
        \)
    \end{center}
    Here we can see now that if we multiply \(b\) by the inverse of \(a\) we get \(x\).
    \item Workout the inverse of \(a\):
    The next step is to workout the inverse of \(a\) so we can then multiply the inverse by \(b\).
    \begin{center}
    
    \(
        \begin{pmatrix}
            3 & 2\\
            5 & 3
        \end{pmatrix}
        ^{-1} = 
        \frac{1}{3*5 - 2*5}
        \cdot
        \begin{pmatrix}
            3 & -2\\
            -5 & 3
        \end{pmatrix}
    \)
    \end{center}

    \begin{center}
        \(
            = \frac{1}{-1}
            \cdot
            \begin{pmatrix}
                3 & -2\\
                -5 & 3
            \end{pmatrix}
            =
            \begin{pmatrix}
                3 & -2\\
                -5 & 3
            \end{pmatrix}
        \)
    \end{center}
    \item Multiply the inverse by \(b\):
    Now that we have the inverse, we multiply it by \(b\) to find \(x\):
    \begin{center}
        \(
            x = a^{-1} \cdot b = 
            \begin{pmatrix}
                -3 & 2\\
                5 & -3
            \end{pmatrix}
            \cdot
            \begin{pmatrix}
                -3\\
                -4
            \end{pmatrix}
        \)

        \(
            =
            \begin{pmatrix}
                -3 * (-3) & 2 * (-4)\\
                5 * (-3) & -3 * (-4)
            \end{pmatrix}
            =
            \begin{pmatrix}
                9 + (-8)\\
                -15 + (-12)
            \end{pmatrix}
            =
            \begin{pmatrix}
                1 \\ -3
            \end{pmatrix}
        \)
    \end{center}

    Here we can see that the \(x\) and \(y\) are \(\begin{pmatrix}1 \\ -3\end{pmatrix}\)
\end{enumerate}
\end{document}